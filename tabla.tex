%
% FECHA: 2021-05-02
% AUTOR: Julio Alejandro Santos Corona
% CORREO: jualesac@yahoo.com
% TÍTULO: tabla.tex
%
% Descripción: Documentación de la clase AJAX.
%

\documentclass[10pt]{article}
\usepackage[spanish]{babel}
\usepackage[utf8]{inputenc}
\usepackage{listings}
\usepackage{anysize}
\usepackage{colortbl}
\title{TABLA v.1.8}
\author{Julio Alejandro Santos Corona}
\date{02 de mayo de 2021}

\marginsize{2.5cm}{2.5cm}{0cm}{1.5cm}

\definecolor{comment}{rgb}{0.5, 0.5, 0.5}
\definecolor{background}{rgb}{0.21, 0.24, 0.25}

\begin{document}
\lstdefinelanguage{JavaScript}{
	morekeywords={
		new,
		function,
		var,
		let,
		limit,
		setConfig,
		getCols,
		getRows,
		getLimit,
		reset,
		clear,
		onEdit,
		onScroll,
		onPagination,
		lines,
		deleteLine
	},
	sensitive=false,
	morecomment=[l][\color{comment}]{//},
	morecomment=[s][\color{comment}]{/*}{*/}
}

\lstset{breaklines=true, tabsize=4, language=JavaScript}
\lstset{basicstyle=\small, numbers=left, numberstyle=\small, stepnumber=1, numbersep=-12pt, backgroundcolor=\color{white}, frame=leftline}

\maketitle

La clase provee los elementos principales para la creación de una tabla a través de una configuración inicial y la definición de comportamiento.

\section{Instancia}

La instancia se crea a través del constructor de la función \textbf{JS\_TABLA.main} al que se le pasan 3 argumentos, width, height, id (default: js\_tabla) del div a utilizar.
\\
\begin{lstlisting}
	var tabla = new JS_TABLA.main ("100%", "400px");
\end{lstlisting}

\subsection{Configuración}

La configuración de la tabla se realiza a través del método \textbf{setConfig} que solicita un argumento tipo object con la siguiente estructura:
\\\\
\begin{tabular}{|m{2.2cm}|m{1.5cm}|m{11.5cm}|}
	\hline
	\rowcolor{black}\textcolor{white}{Parámetro} & \textcolor{white}{Tipo} & \textcolor{white}{Descripción} \\
	\hline
    colnameAlign & boolean & En caso de ser activada, la alineación de los títulos será la misma que la del contenido de cada celda en la columna. \\
    \hline
    limitLines & boolean & Activa la limitación de filas en la tabla, el número límite se define con el método \textbf{limit}. \\
    \hline
    pagination & boolean & Muestra los controles de paginación. \\
    \hline
    callbackScroll & boolean & Activa la posibilidad de ejecutar una petición al llegar al final del scroll de filas (útil para tablas que muestran información particionada). \\ 
    \hline
    columns & object & Configuración de las columnas. \\
    \hline
\end{tabular}
\\\\\\
Columns requiere los objetos por columna con la siguiente estructura:
\\\\
\begin{tabular}{|m{2.2cm}|m{1.5cm}|m{11.5cm}|}
	\hline
	\rowcolor{black}\textcolor{white}{Parámetro} & \textcolor{white}{Tipo} & \textcolor{white}{Descripción} \\
	\hline
	name & string & Nombre de la columna. \\
	\hline
	size & string & Tamaño de la columna; admite los valores aceptados de \textbf{width}, CSS. \\
	\hline
	align & string & Valores admitidos por text-align. \\
	\hline
	id & boolean & Registra el valor de la celda como id para la fila. (Oculta por default). \\
	\hline
	argument & boolean & Guarda el valor de la celda en un objeto que será aprovechado por el callback del método \textbf{lines}. (Oculta por default). \\
	\hline
	visible & boolean & Muestra la columna configurada como id o argumento. \\
	\hline
	edit & boolean & Permite modificar el valor de la celda. \\
	\hline
\end{tabular}
\\
\begin{lstlisting}
	var config = {
		colnameAlign: true,
		limitLines: true,
		pagination: false,
		callbackScroll: true,
		columns: {
			0: { name: "id", id: true },
			1: { name: "nombre", size: "10em" },
			2: { name: "edad", size: "2em", align: "center" },
			3: { name: "salario", size: "5em", align: "right", edit: true },
			4: { name: "estado", size: "3em", argument: true, visible: true }
		}		
	};
\end{lstlisting}

\section{Crear filas}

Las filas pueden provenir de una consulta a base de datos pues la estructura arrojada por la misma cumple sin problema con la estructura siguiente:
\\
\begin{lstlisting}
	{
		0: {
			0: "val1",
			1: "val2",
			2: "val3",
			.
			.
			.
		},
		
		1: {
			0: "val1",
			1: "val2",
			2: "val3",
			.
			.
			.
		},
		.
		.
		.
	}
\end{lstlisting}
\noindent
para imprimir las filas bastará con utilizar el método \textbf{lines}, dicho método solicita 3 argumentos que a continuación se describen:
\\\\
\begin{tabular}{|m{1.8cm}|m{1.5cm}|m{1.5cm}|m{9.5cm}|}
	\hline
	\rowcolor{black}\textcolor{white}{Argumento} & \textcolor{white}{Tipo} & \textcolor{white}{Opcional} & \textcolor{white}{Descripción} \\
	\hline
	data & object & No & Valores a rellenar, el objeto debe contener la estructura antes descrita. \\
	\hline
	callback & function & Si & Función que se ejecutará por cada fila creada, recibe 2 argumentos, \textbf{line} y \textbf{argments}. \\
	\hline
	include & boolean & Si & \textbf{False}(default): La tabla se limpia antes de insertar los valores de data. \textbf{True}: Los valores son agregados inmediatamente abajo de los existentes. \\
	\hline
\end{tabular}
\\\\\\
Ejemplo:
\begin{lstlisting}
	var tabla = new JS_TABLA.main ("100%", "400px");
	let consulta;
	
	tabla.setConfig (config); //Suponiendo que config ya fue asignado
	
	consulta = new Promise (/*Peticion ajax*/);

	consulta.then (function (data) {
		tabla.lines (data);
	});
\end{lstlisting}

\subsection{Utilizando callback en lines}

El método \textbf{lines} tiene la posibilidad de ejecutar una función por cada fila a crear y recibe dos argumentos:
\\\\
\begin{tabular}{|m{1.8cm}|m{2.5cm}|m{10.4cm}|}
	\hline
	\rowcolor{black}\textcolor{white}{Argumento} & \textcolor{white}{Tipo} & \textcolor{white}{Descripción} \\
	\hline
	line & HTMLElement & Objeto HTMLELement que contiene todas las propiedades de la fila. \\
	\hline
	argments & object & En caso de haber configurado columnas como argument este objeto contendrá los valores con el nombre configurado en \textbf{name}. \\
	\hline
\end{tabular}
\\\\\\
adicional las líneas son construidas incluyendo la propiedad tblProperties con los argumentos de la fila.

\newpage
Ejemplo:

\begin{lstlisting}
	let consulta = new Promise (/*Peticion ajax*/);
	
	/*
	 * Este callback valida a traves de un argumento el estatus y coloca un
	 * estilo para toda la fila.
	*/
	consulta.then (function (data) {
		tabla.lines (data, function (line, argments) {
			//Codigo
			if (argments.status == 2) {
				line.style.background = "rgb(250, 0, 0)";
			}
		});
	});
\end{lstlisting}

\section{Eventos}

La clase cuenta con 3 eventos disponibles \textbf{onEdit}, \textbf{onScroll} y \textbf{onPagination}, cada uno requiere previa configuración y solicitan una función como argumento.
\\\\
\begin{tabular}{|m{2cm}|m{2.5cm}|m{10.2cm}|}
	\hline
	\rowcolor{black}\textcolor{white}{Método} & \textcolor{white}{Requerimiento} & \textcolor{white}{Descripción} \\
	\hline
	onEdit & edit & La función se ejecuta al cambiar el valor de una celda configurada como editable. La función no se ejecutará si se oprime la tecla \textbf{Escape} o si el valor editado resulta igual al original. Adicionalmente el callback recibe tres argumentos la línea, la celda modificada (ambos HTMLElement) y el valor anterior. \\
	\hline
	onScroll & callbackScroll & Su ejecución se realizará al llegar al final del scroll. Funcionalidad útil si se requiere cargar datos extra. \\
	\hline
	onPagination & pagination & La función se ejecuta al intentar paginar la tabla, dicha función recibe un objeto como argumento:
		\begin{tabular}{|m{2cm}|m{2cm}|m{4.9cm}|}
			\hline
			Argumento & Tipo & Descripción \\
			\hline
			currentPage & number & Página actual. \\
			\hline
			nextPage & number & Página siguiente luego de la paginación. \\
			\hline
			start & number & Inicio del rango a obtener en la paginación. \\
			\hline
			end & number & Final del rango a obtener en la paginación. \\
			\hline
			rollback & function & Método que activa la paginación sin actualizar página. \\
			\hline
			commit & function & Método que actualiza el número de página. \\
		\end{tabular} \\
	\hline
\end{tabular}

\section{Métodos extra}

\begin{itemize}
	\item \textbf{setLimit}: Configura la cantidad de filas a mostrar; sólo aplica en la configuración general y en la paginación; requiere un argumento number mayor a 0.

	\item \textbf{getLimit}: Obtiene el límite de filas configurado.
	
	\item \textbf{getCols}: Obtiene la cantidad de columnas configuradas.
	
	\item \textbf{getRows}: Obtiene la cantidad de filas mostradas.

	\item \textbf{setDeleteTime}: Configura el tiempo de duración para la eliminación de una fila.
	
	\item \textbf{deleteLine}: Elimina una fila, solicita un argumento que puede ser el HTMLElement del objeto a eliminar o el id de la fila.
	
	\item \textbf{clear}: Limpia las filas mostradas.
	
	\item \textbf{reset}: Elimina la tabla completa, columnas y filas.
\end{itemize}

\end{document}
